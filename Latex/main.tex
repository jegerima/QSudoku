%%%%%%%%%%%%%%%%%%%%%%%%%%%%%%%%%%%%%%%%%
% The Legrand Orange Book
% LaTeX Template
% Version 1.2 (19/5/13)
%
% This template has been downloaded from:
% http://www.LaTeXTemplates.com
%
% Original author:
% Mathias Legrand (legrand.mathias@gmail.com)
%
% License:
% CC BY-NC-SA 3.0 (http://creativecommons.org/licenses/by-nc-sa/3.0/)
%
% Compiling this template:
% This template uses biber for its bibliography and makeindex for its index.
% This means that to update the bibliography and index in this template you
% will need to run the following sequence of commands in the template
% directory:
%
% 1) pdflatex main
% 2) makeindex main.idx -s StyleInd.ist
% 3) biber main
% 4) pdflatex main
%
% This template also uses a number of packages which may need to be
% updated to the newest versions for the template to compile. It is strongly
% recommended you update your LaTeX distribution if you have any
% compilation errors.
%
% Important note:
% Chapter heading images should have a 2:1 width:height ratio,
% e.g. 920px width and 460px height.
%
%%%%%%%%%%%%%%%%%%%%%%%%%%%%%%%%%%%%%%%%%

%----------------------------------------------------------------------------------------
%	PACKAGES AND OTHER DOCUMENT CONFIGURATIONS
%----------------------------------------------------------------------------------------

\documentclass[11pt,fleqn]{book} % Default font size and left-justified equations

\usepackage[top=3cm,bottom=3cm,left=3.2cm,right=3.2cm,headsep=10pt,a4paper]{geometry} % Page margins
\usepackage{float}
\usepackage{xcolor} % Required for specifying colors by name
\definecolor{ocre}{RGB}{243,102,25} % Define the orange color used for highlighting throughout the book

% Font Settings
\usepackage{avant} % Use the Avantgarde font for headings
%\usepackage{times} % Use the Times font for headings
\usepackage{mathptmx} % Use the Adobe Times Roman as the default text font together with math symbols from the Sym­bol, Chancery and Com­puter Modern fonts

\usepackage{microtype} % Slightly tweak font spacing for aesthetics
\usepackage[utf8]{inputenc} % Required for including letters with accents
\usepackage[T1]{fontenc} % Use 8-bit encoding that has 256 glyphs

% Bibliography
\usepackage[style=alphabetic,sorting=nyt,sortcites=true,autopunct=true,babel=hyphen,hyperref=true,abbreviate=false,backref=true,backend=biber]{biblatex}
\addbibresource{bibliography.bib} % BibTeX bibliography file
\defbibheading{bibempty}{}

% Index
\usepackage{calc} % For simpler calculation - used for spacing the index letter headings correctly
\usepackage{makeidx} % Required to make an index
\makeindex % Tells LaTeX to create the files required for indexing

%----------------------------------------------------------------------------------------

\input{structure} % Insert the commands.tex file which contains the majority of the structure behind the template

\begin{document}

%----------------------------------------------------------------------------------------
%	Titulo
%----------------------------------------------------------------------------------------

\begingroup
\thispagestyle{empty}
\AddToShipoutPicture*{\put(6,5){\includegraphics[scale=1]{background}}} % Image background
\centering
\vspace*{9cm}
\par\normalfont\fontsize{35}{35}\sffamily\selectfont
Sudoku\par % Book title
\vspace*{1cm}
{\Huge Jefferson Rivera}\par % Author name
{\Huge Rubén Carvanal}\par % Author name
{\Huge César Madrid}\par % Author name
\endgroup

%----------------------------------------------------------------------------------------
%	COPYRIGHT PAGE
%----------------------------------------------------------------------------------------

\newpage
~\vfill
\thispagestyle{empty}

\noindent Copyright \copyright\ 2013 César Madrid\\ % Copyright notice

\noindent \textsc{Published by Publisher}\\ % Publisher

\noindent \textsc{book-website.com}\\ % URL

\noindent Licensed under the Creative Commons Attribution-NonCommercial 3.0 Unported License (the ``License''). You may not use this file except in compliance with the License. You may obtain a copy of the License at \url{http://creativecommons.org/licenses/by-nc/3.0}. Unless required by applicable law or agreed to in writing, software distributed under the License is distributed on an \textsc{``AS IS'' BASIS, WITHOUT WARRANTIES OR CONDITIONS OF ANY KIND}, either express or implied. See the License for the specific language governing permissions and limitations under the License.\\ % License information

\noindent \textit{Primera edicion, Julio 2013} % Printing/edition date

%----------------------------------------------------------------------------------------
%	TABLE OF CONTENTS
%----------------------------------------------------------------------------------------

\chapterimage{chapter_head_1.pdf} % Table of contents heading image

\pagestyle{empty} % No headers

\tableofcontents % Print the table of contents itself

\cleardoublepage % Forces the first chapter to start on an odd page so it's on the right

\pagestyle{fancy} % Print headers again

%----------------------------------------------------------------------------------------
%	Capitulo 1
%----------------------------------------------------------------------------------------

\chapterimage{chapter_head_2.pdf} % Chapter heading image

\chapter{Introduccion}

\section{Quienes somos?}\index{Paragraphs of Text}

Somos un grupo de tres estudiantes de Ingenieria en Ciencias Computacionales en la Escuela Superior Politecnica del Litoral, trabajamos en en proyecto para la materia Lenguajes de Programacion a cargo del profesor Javier Alejandro Tibau Benitez.


%------------------------------------------------

\section{Que es Sudoku?}\index{Paragraphs of Text}

"Sudoku es un pasatiempo que se publicó por primera vez a finales de la década de 1970 y se popularizó en Japón en 1986, dándose a conocer en el ámbito internacional en 2005 cuando numerosos periódicos empezaron a publicarlo en su sección de pasatiempos.El objetivo del sudoku es rellenar una cuadrícula de 9 x 9 celdas dividida en subcuadriculas de 3 x 3 (también llamadas "cajas" o "regiones") con las cifras del 1 al 9 partiendo de algunos números ya dispuestos en algunas de las celdas.  Aunque se podrían usar colores, letras, figuras, se conviene en usar números para mayor claridad, lo que importa, es que sean nueve elementos diferenciados, que no se deben repetir en una misma fila, columna o subcuadrícula. Un sudoku está bien planteado si la solución es única. La solución de un sudoku siempre es un cuadrado latino, aunque el recíproco en general no es cierto ya que el sudoku establece la restricción añadida de que no se puede repetir un mismo número en una región." \cite{http://es.wikipedia.org/wiki/Sudoku}


%------------------------------------------------

\section{Reglas de Juego}\index{Lists}

El sudoku se compone de 3 reglas basadas en el mismo punto, colocar numeros del 1 al 9 sin que se repitan en filas columnas y subcuadriculas.

Lists are useful to present information in a concise and/or ordered way\footnote{Footnote example...}.

\subsection{Filas}\index{Lists!Numbered List}

\begin{figure}[H]
\centering\includegraphics[scale=0.5]{ejfila1.png}
\caption{ejmeplo de sudoku}
\end{figure}

En la imagen se puede observar que en la primera fila pueden ir los numeros 9 y 3, mientras que en la segunda casilla solo encaja el 7 asi que es un número seguro a llenar

\begin{figure}[H]
\centering\includegraphics[scale=0.5]{ejfila.png}
\caption{ejmeplo de sudoku}
\end{figure}

\subsection{Columnas}\index{Lists!Numbered List}

\begin{figure}[H]
\centering\includegraphics[scale=0.5]{ejcol.png}
\caption{ejmeplo de sudoku}
\end{figure}

En esta imagen se puede observar que en la columna indicada solo puede ir el número 9 asi que es un numero seguro a llenar.

\subsection{Subcuadriculas}\index{Lists!Numbered List}

\begin{figure}[H]
\centering\includegraphics[scale=0.5]{ejsub.png}
\caption{ejmeplo de sudoku}
\end{figure}

En esta imagen observando la subcuadricula se nota que aun faltan 2 numeros por llenar, pero podemos observar que en la primera fila solo quedaun cuadro vacio con lo que aseguramos que ahi va un número 3, lo que nos deja la cuadricula con un espacio, donde cuadra el numero 8.

\begin{figure}[h]
\centering\includegraphics[scale=0.5]{ejsub1.png}
\caption{ejmeplo de sudoku}
\end{figure}


%----------------------------------------------------------------------------------------
%	Capitulo2
%----------------------------------------------------------------------------------------

\chapterimage{chapter_head_2.pdf} % Chapter heading image

\chapter{Arrancar la Aplicacion}

\section{Como abrir la aplicacion.}\index{Paragraphs of Text}

Para abrir nuestro juego debe ubircarse en la carpeta donde lo contiene y dependiendo de sus sistema operativo a abrira el programa indicado.



\subsection{Para windows.}\index{Lists!Numbered List}


En windows tienes que abrir el archivo llamado qsudoku.exe

\subsection{Para linux}\index{Lists!Numbered List}


En Linux arranca el archivo llamado qsudoku.gts




%----------------------------------------------------------------------------------------
%	Capitulo 3
%----------------------------------------------------------------------------------------

\chapterimage{chapter_head_2.pdf} % Chapter heading image

\chapter{Uso de la aplicacion}

\section{Seleccion de niveles}\index{Paragraphs of Text}

Una vez iniciada la aplicacion les va a mostrar las dificultades de juego, desde facil hasta modo leyenda, con las que podras ir demostrando tus habilidades de juego.

Selecciona una difiltudad segun su nivel de confianza en este juego .
\begin{enumerate}
\item Facil para principiantes y gente que nunca ha jugado antes.
\item Intermedio para gente que ya ha mejorado sus tecnicas y puede resolver los faciles mas rapido.
\item Profesional para gente ya con experiencia y ya tiene una habilidad para avanzar a este nivel.
\item Leyenda.- Se dice que hace mucho tiempo alguien logro terminar este sudoku (Juegalo bajo tu propio riesgo).
\end{enumerate}


%------------------------------------------------

\section{Que es Sudoku?}\index{Paragraphs of Text}

"Sudoku es un pasatiempo que se publicó por primera vez a finales de la década de 1970 y se popularizó en Japón en 1986, dándose a conocer en el ámbito internacional en 2005 cuando numerosos periódicos empezaron a publicarlo en su sección de pasatiempos.El objetivo del sudoku es rellenar una cuadrícula de 9 x 9 celdas dividida en subcuadriculas de 3 x 3 (también llamadas "cajas" o "regiones") con las cifras del 1 al 9 partiendo de algunos números ya dispuestos en algunas de las celdas.  Aunque se podrían usar colores, letras, figuras, se conviene en usar números para mayor claridad, lo que importa, es que sean nueve elementos diferenciados, que no se deben repetir en una misma fila, columna o subcuadrícula. Un sudoku está bien planteado si la solución es única. La solución de un sudoku siempre es un cuadrado latino, aunque el recíproco en general no es cierto ya que el sudoku establece la restricción añadida de que no se puede repetir un mismo número en una región." \cite{http://es.wikipedia.org/wiki/Sudoku}


%------------------------------------------------

\section{Reglas de Juego}\index{Lists}

El sudoku se compone de 3 reglas basadas en el mismo punto, colocar numeros del 1 al 9 sin que se repitan en filas columnas y subcuadriculas.

Lists are useful to present information in a concise and/or ordered way\footnote{Footnote example...}.

\subsection{Filas}\index{Lists!Numbered List}

\begin{figure}[H]
\centering\includegraphics[scale=0.5]{ejfila1.png}
\caption{ejmeplo de sudoku}
\end{figure}

En la imagen se puede observar que en la primera fila pueden ir los numeros 9 y 3, mientras que en la segunda casilla solo encaja el 7 asi que es un número seguro a llenar

\begin{figure}[H]
\centering\includegraphics[scale=0.5]{ejfila.png}
\caption{ejmeplo de sudoku}
\end{figure}

\subsection{Columnas}\index{Lists!Numbered List}

\begin{figure}[H]
\centering\includegraphics[scale=0.5]{ejcol.png}
\caption{ejmeplo de sudoku}
\end{figure}

En esta imagen se puede observar que en la columna indicada solo puede ir el número 9 asi que es un numero seguro a llenar.

\subsection{Subcuadriculas}\index{Lists!Numbered List}

\begin{figure}[H]
\centering\includegraphics[scale=0.5]{ejsub.png}
\caption{ejmeplo de sudoku}
\end{figure}

En esta imagen observando la subcuadricula se nota que aun faltan 2 numeros por llenar, pero podemos observar que en la primera fila solo quedaun cuadro vacio con lo que aseguramos que ahi va un número 3, lo que nos deja la cuadricula con un espacio, donde cuadra el numero 8.

\begin{figure}[h]
\centering\includegraphics[scale=0.5]{ejsub1.png}
\caption{ejmeplo de sudoku}
\end{figure}


%----------------------------------------------------------------------------------------
%	CHAPTER 2
%----------------------------------------------------------------------------------------

\chapter{In-text Elements}

\section{Theorems}\index{Theorems}

This is an example of theorems.

\subsection{Several equations}\index{Theorems!Several Equations}

\begin{theorem}
In $E=\mathbb{R}^n$ all norms are equivalent. It has the properties:
\begin{align}
& \big| ||\mathbf{x}|| - ||\mathbf{y}|| \big|\leq || \mathbf{x}- \mathbf{y}||\\
&  ||\sum_{i=1}^n\mathbf{x}_i||\leq \sum_{i=1}^n||\mathbf{x}_i||\quad\text{where $n$ is a finite integer}
\end{align}
\end{theorem}

\subsection{Single Line}\index{Theorems!Single Line}

\begin{theorem}
A set $\mathcal{D}(G)$ in dense in $L^2(G)$, $|\cdot|_0$. 
\end{theorem}

%------------------------------------------------

\section{Definitions}\index{Definitions}

This is an example of a definition. A definition could be mathematical or it could define a concept.

\begin{definition}[Definition name]
Given a vector space $E$, a norm on $E$ is an application, denoted $||\cdot||$, $E$ in $\mathbb{R}^+=[0,+\infty[$ such that:
\begin{align}
& ||\mathbf{x}||=0\ \Rightarrow\ \mathbf{x}=\mathbf{0}\\
& ||\lambda \mathbf{x}||=|\lambda|\cdot ||\mathbf{x}||\\
& ||\mathbf{x}+\mathbf{y}||\leq ||\mathbf{x}||+||\mathbf{y}||
\end{align}
\end{definition}

%------------------------------------------------

\section{Notations}\index{Notations}

\begin{notation}
Given an open subset $G$ of $\mathbb{R}^n$, the set of functions $\varphi$ are:
\begin{enumerate}
\item Bounded support $G$;
\item Infinitely differentiable;
\end{enumerate}
a vector space is denoted by $\mathcal{D}(G)$. 
\end{notation}

%------------------------------------------------

\section{Remarks}\index{Remarks}

This is an example of a remark.

\begin{remark}
The concepts presented here are now in conventional employment in mathematics. Vector spaces are taken over the field $\mathbb{K}=\mathbb{R}$, however, established properties are easily extended to $\mathbb{K}=\mathbb{C}$.
\end{remark}

%------------------------------------------------

\section{Corollaries}\index{Corollaries}

This is an example of a corollary.

\begin{corollary}[Corollary name]
The concepts presented here are now in conventional employment in mathematics. Vector spaces are taken over the field $\mathbb{K}=\mathbb{R}$, however, established properties are easily extended to $\mathbb{K}=\mathbb{C}$.
\end{corollary}

%------------------------------------------------

\section{Propositions}\index{Propositions}

This is an example of propositions.

\subsection{Several equations}\index{Propositions!Several Equations}

\begin{proposition}[Proposition name]
It has the properties:
\begin{align}
& \big| ||\mathbf{x}|| - ||\mathbf{y}|| \big|\leq || \mathbf{x}- \mathbf{y}||\\
&  ||\sum_{i=1}^n\mathbf{x}_i||\leq \sum_{i=1}^n||\mathbf{x}_i||\quad\text{where $n$ is a finite integer}
\end{align}
\end{proposition}

\subsection{Single Line}\index{Propositions!Single Line}

\begin{proposition} 
Let $f,g\in L^2(G)$; if $\forall \varphi\in\mathcal{D}(G)$, $(f,\varphi)_0=(g,\varphi)_0$ then $f = g$. 
\end{proposition}

%------------------------------------------------

\section{Examples}\index{Examples}

This is an example of examples.

\subsection{Equation and Text}\index{Examples!Equation and Text}

\begin{example}
Let $G=\{x\in\mathbb{R}^2:|x|<3\}$ and denoted by: $x^0=(1,1)$; consider the function:
\begin{equation}
f(x)=\left\{\begin{aligned} & \mathrm{e}^{|x|} & & \text{si $|x-x^0|\leq 1/2$}\\
& 0 & & \text{si $|x-x^0|> 1/2$}\end{aligned}\right.
\end{equation}
The function $f$ has bounded support, we can take $A=\{x\in\mathbb{R}^2:|x-x^0|\leq 1/2+\epsilon\}$ for all $\epsilon\in\intoo{0}{5/2-\sqrt{2}}$.
\end{example}

\subsection{Paragraph of Text}\index{Examples!Paragraph of Text}

\begin{example}[Example name]
\lipsum[2]
\end{example}

%------------------------------------------------

\section{Exercises}\index{Exercises}

This is an example of an exercise.

\begin{exercise}
This is a good place to ask a question to test learning progress or further cement ideas into students' minds.
\end{exercise}

%------------------------------------------------

\section{Problems}\index{Problems}

\begin{problem}
What is the average airspeed velocity of an unladen swallow?
\end{problem}

%------------------------------------------------

\section{Vocabulary}\index{Vocabulary}

Define a word to improve a students' vocabulary.

\begin{vocabulary}[Word]
Definition of word.
\end{vocabulary}

%----------------------------------------------------------------------------------------
%	CHAPTER 3
%----------------------------------------------------------------------------------------

\chapterimage{chapter_head_1.pdf} % Chapter heading image

\chapter{Presenting Information}

\section{Table}\index{Table}

\begin{table}[h]
\centering
\begin{tabular}{l l l}
\toprule
\textbf{Treatments} & \textbf{Response 1} & \textbf{Response 2}\\
\midrule
Treatment 1 & 0.0003262 & 0.562 \\
Treatment 2 & 0.0015681 & 0.910 \\
Treatment 3 & 0.0009271 & 0.296 \\
\bottomrule
\end{tabular}
\caption{Table caption}
\end{table}

%------------------------------------------------

\section{Figure}\index{Figure}

\begin{figure}[h]
\centering\includegraphics[scale=0.5]{ejsub.png}
\caption{ejmeplo de sudoku}
\end{figure}

%----------------------------------------------------------------------------------------
%	BIBLIOGRAPHY
%----------------------------------------------------------------------------------------

\chapter*{Bibliography}
\addcontentsline{toc}{chapter}{\textcolor{ocre}{Bibliography}}
\section*{Books}
\addcontentsline{toc}{section}{Books}
\printbibliography[heading=bibempty,type=book]
\section*{Articles}
\addcontentsline{toc}{section}{Articles}
\printbibliography[heading=bibempty,type=article]

%----------------------------------------------------------------------------------------
%	INDEX
%----------------------------------------------------------------------------------------

\cleardoublepage
\setlength{\columnsep}{0.75cm}
\addcontentsline{toc}{chapter}{\textcolor{ocre}{Index}}
\printindex

%----------------------------------------------------------------------------------------

\end{document}